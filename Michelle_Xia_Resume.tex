% Options for packages loaded elsewhere
\PassOptionsToPackage{unicode}{hyperref}
\PassOptionsToPackage{hyphens}{url}
%
\documentclass[
]{article}
\usepackage{amsmath,amssymb}
\usepackage{iftex}
\ifPDFTeX
  \usepackage[T1]{fontenc}
  \usepackage[utf8]{inputenc}
  \usepackage{textcomp} % provide euro and other symbols
\else % if luatex or xetex
  \usepackage{unicode-math} % this also loads fontspec
  \defaultfontfeatures{Scale=MatchLowercase}
  \defaultfontfeatures[\rmfamily]{Ligatures=TeX,Scale=1}
\fi
\usepackage{lmodern}
\ifPDFTeX\else
  % xetex/luatex font selection
\fi
% Use upquote if available, for straight quotes in verbatim environments
\IfFileExists{upquote.sty}{\usepackage{upquote}}{}
\IfFileExists{microtype.sty}{% use microtype if available
  \usepackage[]{microtype}
  \UseMicrotypeSet[protrusion]{basicmath} % disable protrusion for tt fonts
}{}
\makeatletter
\@ifundefined{KOMAClassName}{% if non-KOMA class
  \IfFileExists{parskip.sty}{%
    \usepackage{parskip}
  }{% else
    \setlength{\parindent}{0pt}
    \setlength{\parskip}{6pt plus 2pt minus 1pt}}
}{% if KOMA class
  \KOMAoptions{parskip=half}}
\makeatother
\usepackage{xcolor}
\setlength{\emergencystretch}{3em} % prevent overfull lines
\providecommand{\tightlist}{%
  \setlength{\itemsep}{0pt}\setlength{\parskip}{0pt}}
\setcounter{secnumdepth}{-\maxdimen} % remove section numbering
\usepackage{bookmark}
\IfFileExists{xurl.sty}{\usepackage{xurl}}{} % add URL line breaks if available
\urlstyle{same}
\hypersetup{
  hidelinks,
  pdfcreator={LaTeX via pandoc}}

\author{}
\date{}

\begin{document}

\textbf{Michelle Xia}

(720) 445-6119 {\emph{\textbar{}} }michellexia2012@gmail.com{
\emph{\textbar{}} }{linkedin.com/in/michelle-xia}

\hfill\break

{\textbf{\emph{WORK EXPERIENCE}}}

\textbf{Vail Resorts{ }}November 2023 -- April 2024

\emph{Ski Instructor {[}Hiatus from software engineering{]}{ }}Vail, CO

\hfill\break

\textbf{Amazon}{\textbf{\emph{{ }}}}August 2021 {\emph{--} }August 2023

\emph{Software Development Engineer}{\emph{{ }}}Denver, CO

\begin{itemize}
\tightlist
\item
  Designed a new \textbf{RESTful microservice} adding cloud
  functionality to an existing service to enable users to update app
  configurations to device test groups with less than \textbf{5 ms per
  call} using \textbf{AWS} \textbf{ECS}, \textbf{Route53}, \textbf{ELB},
  and \textbf{Java}{~}
\item
  \textbf{Implemented and deployed} a microservice connecting to an
  existing service with \textbf{1 million} transactions per minute
  without affecting existing service traffic and maintaining existing
  SLAs
\item
  Designed and deployed \textbf{CI/CD pipelines} using
  \textbf{TypeScript} and \textbf{CloudFormation} to minimize deployment
  risks
\item
  Developed \textbf{Python} script to research metric patterns,
  \textbf{saving \textasciitilde\$100k yearly} by enabling re-allocation
  of resources
\item
  Architected and built website using \textbf{Angular JS}, \textbf{AWS
  CloudFront}, \textbf{CloudWatch}, \textbf{CSRF tokens}, \textbf{CORS},
  \textbf{SSL}, and \textbf{Midway Token Auth} to migrate the existing
  website to new AWS infrastructure using new auth
\item
  Designed and developed back-end logic in \textbf{Java} for processing
  device engagement metrics on new device launch
\item
  Developed rigorous \textbf{unit tests} with several edge cases on new
  device engagement metrics to ensure code integrity
\end{itemize}

\hfill\break

\textbf{University of Texas at Austin}{\emph{{ }September 202}}0{
\textbf{\emph{}} \emph{--} }May 2021

\emph{Research Assistant}{\emph{{ }}}Austin, TX

\begin{itemize}
\tightlist
\item
  Demonstrated \textbf{image classification} recommender systems
  outperform state-of-the-art text recommender systems{~}
\item
  Recipient of the University of Texas at Austin Undergraduate Research
  Fellow award{~}
\end{itemize}

\hfill\break

\textbf{Cisco Systems}{\emph{{ }}}May 2020{\emph{-- August 20}}20

\emph{Security Software Engineer Intern}{\emph{{ }}}Richardson, TX

\begin{itemize}
\tightlist
\item
  Harnessed \textbf{GitHub}, \textbf{Java}, and \textbf{Python}
  technologies within the Software Development Life Cycle process{~}
\item
  Developed solution automating \textbf{threat containment} within a
  corporate network environment to protect client assets{~}
\item
  Implemented an authentication scheme incorporating \textbf{machine
  learning} to perform authentication via facial recognition in order to
  ensure tight security of customers' networks{~}
\item
  Repaired bugs within Cisco code and developed unit tests to streamline
  integrity checking processes of codebases{~}
\end{itemize}

\hfill\break

\textbf{Tyson Foods}{\emph{{ }}}May 2019 {\emph{--} }August 2019

\emph{Cybersecurity Intern}{\emph{{ }}}Springdale, AR

\begin{itemize}
\tightlist
\item
  Built an app converting \textbf{CVSS risk scores} to Tyson-specific
  risk scores, enabling quick, informed decision making
\item
  Scripted risk score enhancement using Twitter \textbf{scraping} and
  \textbf{APIs} to improve \textbf{vulnerability management}
  accuracy\textbf{{~}}
\item
  Created and automated reporting and dashboards while ensuring data
  integrity, improving security posture visibility {~ ~}
\end{itemize}

\hfill\break

\textbf{PROJECTS}

\textbf{Texas A\&M Datathon (Major League Hacking Event)} \emph{- For
You Page}{ }Fall 2020

Technologies used: nltk, spaCy, jupyter, python, numpy, matplotlib,
pandas, gensim, pickle, Streamlit{~}

\begin{itemize}
\tightlist
\item
  Won 1\textsuperscript{st} place out of 1,300 people in the world's
  first and only Data Science Major League Hacking hackathon
\item
  Built a personalized website for next year's Datathon participants
  with creative data gathering methods and NLP
\item
\end{itemize}

\textbf{MIS 333K Bi-Annual Web Application Competition} \emph{- Movie
Theater Website}{\emph{{ }}}Fall 2020

Technologies used: C\#, ASP.NET MVC, Bootstrap CSS

\begin{itemize}
\tightlist
\item
  Won 1\textsuperscript{st} of 55 groups in ConocoPhillips competition
  to design and build movie theater website ({Github Link}){~}
\end{itemize}

\hfill\break

\textbf{Towards Data Science} \emph{- How College Students are Handling
COVID-19}{ }Fall 2020

\begin{itemize}
\tightlist
\item
  Published {article} using Natural Language Processing techniques to
  analyze college students' response to COVID-19
\end{itemize}

\hfill\break

\textbf{EDUCATION}

\textbf{The University of Texas at Austin}{\textbf{\emph{{ }}}}August
2017 -- {\emph{May 202}}1

{\emph{Bachelor of} }\emph{Business Administration,}{ \emph{in}
}\emph{Management Information Systems}{\emph{{ }}}Austin, TX

\begin{itemize}
\tightlist
\item
  Overall GPA: 3.73/4.00
\item
  Certificate in Elements of Computing {[}6 courses in Department of
  Computer Science{]}
\end{itemize}

\hfill\break

{\textbf{\emph{SKILLS \& ADDITIONAL INFORMATION}}}

\textbf{Technical Skills}: Python, Java, AWS, REST API, TypeScript,
AngularJS, NLP, NumPy, Pandas, ASP.NET MVC, SQL

{\textbf{GitHub}: }{https://github.com/michelle-xia}{{~}}

\textbf{Work Eligibility}: Eligible to work in the US with no
restrictions (US Citizen)

\end{document}
